\chapter{Model and Assumptions}
Introduce concept of LTI system here
\section{State Space Model}
We will consider a combined deterministic-stochastic LTI system written in innovation form as
\begin{subequations}\label{eq:2_innovation}
\begin{equation}x(k+1) = Ax(k) + Bu(k) + Ke(k)\end{equation}
\begin{equation}y(k) = Cx(k) + Du(k) + e(k)\end{equation}
\end{subequations}
where $x_k \in \mathbb{R}^n$ is the system state, $u_k \in \mathbb{R}^m$ is the system input, $y_k \in \mathbb{R}^l$ is the system output, and $e_k \in \mathbb{R}^l$ is the innovation. $A$, $B$, $C$, and $D$ are the system matrices with appropriate dimensions and $K$ is the Kalman filter gain. The system represented in (\ref{eq:2_innovation}) can also be represented in predictor form as
\begin{subequations}\label{eq:2_process}
\begin{equation}x(k+1) = A_Kx(k) + B_Ku(k) + Ky(k)\end{equation}
\begin{equation}y(k) = Cx(k) + Du(k) + e(k)\end{equation}
\end{subequations}
where $A_K = A-KC$ and $B_K = B-KD$.

The systems represented by (\ref{eq:2_innovation}) and (\ref{eq:2_process}) are equivalent from an input/output point of view, but because $A_K$ is guaranteed stable even if the original process matrix $A$ is unstable, the predictor form proves advantageous when considering unstable open-loop systems. We will use the state space model in innovation form to derive the general subspace algorithm for identifying combined deterministic-stochastic LTI systems but will rely on the prediction form of the model when considering identification of closed-loop systems.

\section{Assumptions}

\textbf{[Assumption 1]:} $A_K = A - KC$ is stable (i.e. its eigenvalues lie within the unit circle)

\noindent \textbf{[Assumption 2]:} The system is represented in its minimal form
